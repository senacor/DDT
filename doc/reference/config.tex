\documentclass[
   12pt,                % Schriftgroesse 12pt
   a4paper,             % Layout fuer Din A4
   oneside,             % Layout fuer einseitigen Druck
   headinclude,         % Kopfzeile wird Seiten-Layouts mit beruecksichtigt
   %headsepline,         % horizontale Linie unter Kolumnentitel
   %plainheadsepline,    % horizontale Linie auch beim plain-Style
   BCOR12mm,            % Korrektur fuer die Bindung
   DIV18,               % DIV-Wert fuer die Erstellung des Satzspiegels, siehe scrguide
   halfparskip,         % Absatzabstand statt Absatzeinzug
   openany,             % Kapitel k�nnen auf geraden und ungeraden Seiten beginnen
   bibtotoc,            % Literaturverz. wird ins Inhaltsverzeichnis eingetragen
   pointlessnumbers,    % Kapitelnummern immer ohne Punkt
   tablecaptionabove,   % korrekte Abstaende bei TabellenUEBERschriften
   titlepage,			% ja bitte titelei
   nochapterprefix,
   appendixprefix,
   fleqn,               % fleqn: Glgen links (statt mittig)
   %draft               % Keine Bilder in der Anzeige, overfull hboxes werden angezeigt
     ]{scrartcl}

%\usepackage[onehalfspacing]{setspace}
\makeatletter
%\renewcommand*{\chapterheadstartvskip}{%
%	  {\setlength{\@tempdima}{\f@baselineskip}%
%	  \vspace*{2.3\@tempdima}}%
%	}
%\renewcommand*{\chapterheadendvskip}{%
%	  {\setlength{\@tempdima}{\f@baselineskip}%
%	    \vspace{1.725\@tempdima
%	      \@plus .115\@tempdima \@minus .192\@tempdima}}%
%	}
%\makeatother
%\usepackage{ngerman}             % neue Rechtschreibung
%\usepackage[ansinew]{inputenc}  % Input-Encodung: ansinew fuer Windows
\usepackage[latin1]{inputenc}    % Input-Encodung: latin1 fuer Unix
\usepackage[T1]{fontenc}         % T1-kodierte Schriften, korrekte Trennmuster fuer Worte mit Umlauten
%\usepackage{caption}      % mehrzeilige Captions ausrichten
\usepackage{url} 				 % fuer was auch immer
\usepackage[centertags]{amsmath} % AMS-Mathematik, centertags zentriert Nummer bei split
%\usepackage{latexsym}           % verschiedene Symbole
%\usepackage{textcomp}           % verschiedene Symbole
     
%\usepackage[pdftex]{graphicx}            % zum Einbinden von Grafiken
\usepackage{graphicx}            % zum Einbinden von Grafiken
\usepackage{float}               % u.a. genaue Plazierung von Gleitobjekten mit H

\usepackage{scrpage2}            % Kopf und Fusszeilen-Layout 
\renewcommand{\headfont}{\normalfont\sffamily}    % Kolumnentitel serifenlos
\renewcommand{\pnumfont}{\normalfont\sffamily}    % Seitennummern serifenlos
\pagestyle{scrheadings}

% kopfzeile - links mitte rechts
\ihead[]{DDT Reference Documentation}              % Kolumnentitel immer oben innen
\chead[]{}
\ohead[\pagemark]{\pagemark}     % Seitennummern immer oben aussen
% fusszeile - links mitte rechts.
\ifoot[]{}
\cfoot[]{}
\ofoot[]{}                       % Seitennummern in der Fusszeile loeschen
\rohead{ \includegraphics[scale=0.5]{image/logo.pdf}}
\rofoot{\pagemark}
%\setheadwidth{textwithmarginpar}
%\setfootwidth{textwithmarginpar}
%\newcommand{\titleRule}{\rule{\linewidth}{0.5mm}}


\reversemarginpar

% Schrift mit Serifen auch fuer Ueberschriften benutzen
%\renewcommand*{\sectfont}{\bfseries}
%\renewcommand*{\descfont}{\bfseries}

% define the wanted font for all highlightings here
\def\lstbasicfont{\fontfamily{pcr}\selectfont}

%% for sourcecode we use listings-package
\usepackage[usenames,dvipsnames]{color}
\usepackage{listings}
\definecolor{senacorlight}{rgb}{0.96,0.945,0.90} % original Senacor background color
\lstloadlanguages{Java}
\lstset{% general command to set parameter(s)+
basicstyle={\lstbasicfont\footnotesize},
%basicstyle={\ttfamily\small}, % print whole listing small
keywordstyle=\bfseries, % bold black keywords
basewidth=0.51em,
identifierstyle=\color{Black}, % gray
commentstyle=\color{Black}\itshape, % white comments
stringstyle=\color{Black}, % typewriter type for strings
showstringspaces=false, % no special string spaces
language=Java,
tabsize=2,
backgroundcolor=\color{senacorlight}
}

%\usepackage[intoc,german]{nomencl}
%\usepackage[german]{nomencl}
%\renewcommand{\nomname}{Glossar}
%\makenomenclature
%%% some defines

\typearea[current]{current}        % Neuberechnung des Satzspiegels mit alten Werten nach �nderung von Zeilenabstand,etc

%\usepackage{amsart}

\usepackage{ae}                  % F�r PDF-Erstellung
\usepackage{pdfsync}
\usepackage[pdftex,a4paper,
            pdftitle={DDT Reference Documentation},
			pdfauthor={Carl-Eric Menzel},
            colorlinks=false,
            bookmarks=true,
            bookmarksnumbered=true]{hyperref}

